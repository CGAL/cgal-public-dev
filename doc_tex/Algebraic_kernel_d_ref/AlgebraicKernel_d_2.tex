\begin{ccRefConcept}{AlgebraicKernel_d_2} 

\ccDefinition

A model of the \ccc{AlgebraicKernel_d_2} concept gathers necessary tools  
for solving and handling bivariate polynomial systems of general degree $d$.

%The AlgebraicKernel_d_2 concept is meant to provide the curved kernel with all 
%the algebraic functionalities on bivariate polynomials required for the manipulation 
%of arcs of algebraic curves of general degree $d$ in $\R^2$. 

\ccRefines
\ccc{AlgebraicKernel_d_1}
% \ccc{DefaultConstructible}\\ it may be not !
\ccc{CopyConstructible}\\
\ccc{Assignable}\\

\ccTypes
\ccSetTwoColumns{xxxxxxxxxxxxxxxxxxxxxxxxxxxxxxxxxxxxxxxxxxx}{}

A model of \ccc{AlgebraicKernel_d_2} must provide

\ccNestedType{Polynomial_2}{
A model of \ccc{Polynomial_2}, 
where \ccc{CGAL::Polynomial_traits_d<Polynomial_2>::Innermost_coefficient}
is \ccc{AlgebraicKernel_d_1::Coefficient} 
} 

\ccNestedType{Algebraic_real_2}{A model of \ccc{AlgebraicKernel_d_2::AlgebraicReal_2}, for solutions of systems
of two bivariate polynomials of type \ccc{Polynomial_2}.}

\ccHeading{Functors}
\ccNestedType{Construct_algebraic_real_2}{A model of \ccc{AlgebraicKernel_d_2::ConstructAlgebraicReal_2}}\ccGlue
\ccNestedType{Compute_polynomial_x_2}{\ccc{AlgebraicKernel_d_2::ComputePolynomialX_2}}\ccGlue
\ccNestedType{Compute_polynomial_y_2}{\ccc{AlgebraicKernel_d_2::ComputePolynomialY_2}}\ccGlue
\ccNestedType{Compute_isolating_interval_x_2}{\ccc{AlgebraicKernel_d_2::ComputeIsolatingIntervalX_2}}\ccGlue
\ccNestedType{Compute_isolating_interval_y_2}{\ccc{AlgebraicKernel_d_2::ComputeIsolatingIntervalY_2}}\ccGlue
\ccNestedType{Is_square_free_2}{A model of \ccc{AlgebraicKernel_d_2::IsSquareFree_2}.}
\ccNestedType{Make_square_free_2}{A model of \ccc{AlgebraicKernel_d_2::MakeSquareFree_2}.}
\ccNestedType{Square_free_factorize_2}{A model of \ccc{AlgebraicKernel_d_2::SquareFreeFactorize_2}.}
\ccNestedType{Is_coprime_2}{A model of \ccc{AlgebraicKernel_d_2::IsCoprime_2}.}
\ccNestedType{Make_coprime_2}{A model of \ccc{AlgebraicKernel_d_2::MakeCoprime_2}.}
\ccNestedType{Solve_2}{A model of \ccc{AlgebraicKernel_d_2::Solve_2}.} 
\ccNestedType{Sign_at_2}{A model of \ccc{AlgebraicKernel_d_2::SignAt_2}.}
\ccNestedType{Get_x_2}{A model of \ccc{AlgebraicKernel_d_2::GetX_2}.}
\ccNestedType{Get_y_2}{A model of \ccc{AlgebraicKernel_d_2::GetY_2}.} 
\ccNestedType{Compare_x_2}{A model of \ccc{AlgebraicKernel_d_2::CompareX_2}.}
\ccNestedType{Compare_y_2}{A model of \ccc{AlgebraicKernel_d_2::CompareY_2}.}
\ccNestedType{Compare_xy_2}{A model of \ccc{AlgebraicKernel_d_2::CompareXY_2}.}
\ccNestedType{Bound_between_x_2}{A model of \ccc{AlgebraicKernel_d_2::BoundBetweenX_2}.}
\ccNestedType{Bound_between_y_2}{A model of \ccc{AlgebraicKernel_d_2::BoundBetweenY_2}.}
\ccNestedType{Approximate_absolute_x_2}{A model of \ccc{AlgebraicKernel_d_2::ApproximateAbsoluteX_2}}
\ccNestedType{Approximate_absolute_y_2}{A model of \ccc{AlgebraicKernel_d_2::ApproximateAbsoluteY_2}}
\ccNestedType{Approximate_relative_x_2}{A model of \ccc{AlgebraicKernel_d_2::ApproximateRelativeX_2}}
\ccNestedType{Approximate_relative_y_2}{A model of \ccc{AlgebraicKernel_d_2::ApproximateRelativeY_2}}


\ccNestedType{X_critical_points_2}{This is an optional functor. 
	Hence it is either a model of 
	\ccc{AlgebraicKernel_d_2::XCriticalPoints_2} 
	or set to \ccc{CGAL::Null_functor}.}

\ccNestedType{Y_critical_points_2}{This is an optional functor. 
	Hence it is either a model of 
	\ccc{AlgebraicKernel_d_2::YCriticalPoints_2} 
	or set to \ccc{CGAL::Null_functor}.}

%\ccHasModels

%Algebraic_curve_kernel_2

\ccOperations

For each of the function objects above, there must exist a member function that requires no arguments and returns an instance of that function object. The name of the member function is the uncapitalized name of the type returned with the suffix \ccc{_object} appended. For example, for the function object  \ccc{AlgebraicKernel_d_2::BoundBetweenX_2} the following member function must exist:

\ccCreationVariable{ak_2}
\ccMemberFunction{AlgebraicKernel_d_2::BoundBetweenX_2 bound_between_x_2_object() const;}{}

\ccSeeAlso

\ccRefIdfierPage{AlgebraicKernel_d_1}\\

\end{ccRefConcept}

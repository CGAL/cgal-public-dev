\documentclass[11pt]{article}
%\documentclass{svmult}
%% \usepackage[bottom]{footmisc}
\usepackage{epsfig,color,amsmath,amssymb,latexsym}
\usepackage{wrapfig}
\usepackage{multirow}
\usepackage{fancybox}
\usepackage{path}
\usepackage{tabularx}
\usepackage{calc}
\usepackage{url}
%% \usepackage{setspace}

\setlength{\textwidth}{6.3in}
\setlength{\textheight}{9.3in}
\setlength{\evensidemargin}{0.0in}
\setlength{\oddsidemargin}{0.0in}
\setlength{\topmargin}{-0.5in}
\setlength{\parskip}{1.1mm}
\setlength{\baselineskip}{1.7\baselineskip}

\def\email#1{{\tt#1}}

\newtheorem{theorem}{Theorem}
\newtheorem{lemma}[theorem]{Lemma}
\newtheorem{proposition}[theorem]{Proposition}
\newtheorem{observation}[theorem]{Observation}
\newtheorem{corollary}[theorem]{Corollary}
\newtheorem{definition}[theorem]{Definition}

\newenvironment{proof}[1][Proof:]{\begin{trivlist}
\item[\hskip \labelsep {\bfseries #1}]}{\end{trivlist}}

\newcommand{\parms}{{\rm I\!\hspace{-0.025em} P}}
\newcommand{\reals}{\mathbb{R}}
\newcommand{\rrr}{\reals^3}
\newcommand{\rr}{\reals^2}
\newcommand{\spheretwo}{\S^2}
\newcommand{\vecd}{\vec{d}}
\newcommand{\wmax}{w_{\rm max}}
\newcommand{\hfa}{\frac{\alpha}{2}}
\newcommand{\hfp}{\frac{\pi}{2}}
\newcommand{\prm}{{\sc Prm}}
\newcommand{\cgal}{{\sc Cgal}}
\newcommand{\sgal}{{\sc Sgal}}
\newcommand{\lsl}{{\sc Lsl}}
\newcommand{\naive}{na\"{\i}ve}
\newcommand{\Astar}{${\rm A}^{*}$}
\newcommand{\vrml}{{\sc Vrml}}
\newcommand{\boost}{{\sc Boost}}
\newcommand{\dcel}{{\sc Dcel}}

\definecolor{orange}{rgb}{1,0.5,0}
\newcommand{\redpart}{{\sc \textcolor{red}{$R$}}}
\newcommand{\greenpart}{{\sc \textcolor{green}{$G$}}}
\newcommand{\bluepart}{{\sc \textcolor{blue}{$B$}}}
\newcommand{\purplepart}{{\sc \textcolor{magenta}{$P$}}}
\newcommand{\yellowpart}{{\sc \textcolor{orange}{$Y$}}}
\newcommand{\turquoisepart}{{\sc \textcolor{cyan}{$T$}}}

\def\ccode#1{{\small{\texttt{#1}}}}
\newcommand{\aos}{\ccode{Arrangement\_on\_surface\_2}}
\newcommand{\arr}{\ccode{Arrangement\_2}}

\def\eps{{\varepsilon}}
\def\A{{\cal A}}
\def\P{{\cal P}}
\def\dP{\partial{\cal P}}
\def\Q{{\cal Q}}
\def\dQ{\partial{\cal Q}}
\def\S{{\cal S}}

% ======== Comments ===========================================================
\def\marrow{\marginpar[\hfill$\longrightarrow$]{$\longleftarrow$}}
\def\efi#1{{\sc Efi says: }{\marrow\sf #1}}
\def\danny#1{{\sc Danny says: }{\marrow\sf #1}}
\def\asaf#1{{\sc Asaf says: }{\marrow\sf #1}}

% =============================================================================
\title{Lines Through Segments In Three Dimensional Space
  \thanks{This work has been supported in part by the Israel Science
    Foundation (Grant no. 236/06) and by the Hermann Minkowski--Minerva
    Center for Geometry at Tel-Aviv University.}}
% \titlerunning{Computing}
\author{Asaf Porat\thanks{School of Computer Science, Tel-Aviv
  University, 69978, Israel. \email{\{efif,danha\}@post.tau.ac.il}}
  \and Efi Fogel$^{\dagger}$
  \and Dan Halperin$^{\dagger}$
}
%% \date{}
% If the authors are from one institution, do not use the footnotes.
% The \institute command below will not produce footnotes if there is
% only one institution.
%
%% \author{%
%%   Asaf Porat \and
%%   Efi Fogel \and
%%   Dan Halperin
%% }
% Use \authorrunning{Short Authors} for an abbreviated version the
% author list if the complete one is too long
%
%% \institute{%
%%   The Blavatnik School of Computer Science, Tel-Aviv University, 69978, 
%%   Israel,
%%   \texttt{\{efif,danha\}@post.tau.ac.il}
%% }
% =============================================================================
\begin{document}
\maketitle
% =============================================================================
\begin{abstract}
The method described in this paper computes all lines tangent to four line 
segments taken from a set of \textit{n} line segments in three dimensional 
Euclidean space.
The problem of finding all tangent lines to four line segments arises in many
fields of computation such as computer graphics (visibility computations), 
computer geometry (line transversal and assembly planning) and computer 
vision.\newline
The number of tangent lines to four line segments is zero to four including or
infinite. We present an exact implementation of an efficient output sensitive
algorithm.\newline
Theoretical bounds for the algorithm are $O((n^3 + P)log n)$ running time, and
$O(n^2 + P)$ working storage. Where n is the input size and $P$ is the output
size ($P$ is bounded by $n^4$).\newline
We do not assume general position. Namely, the algorithm and its implementation
are robust. The code properly handles all degenerate cases, a line segment may
be degenerate to a point, several segments may intersect, be co-planer, 
parallel, concurrent, on the same supporting line, or even overlap.

\end{abstract}
% =============================================================================
\section{Introduction}
The set of lines tangent to 1 line segment has 3 degrees of freedom, those
tangent to 2 line segments has 2 degrees of freedom, those tangent to 3 has 1 
degree of freedom.\newline
The number of lines tangent to 4 line segments in $\mathbb{R}^3$ is 0, 1, 2 or 
infinite. The problem of finding all line transversals was studied by H. 
Bronnimann, et al. \cite{springerlink:10.1007/s00454-005-1183-1}.
They showed that the number of lines tangent to four arbitrary line segments 
in $\mathbb{R}^3$ is zero to four including or infinite. More precisely, the 
number of lines tangent to four line segments is zero to four, unless the 
lines lie in one of the following configurations:

\begin{enumerate}
\item
The four line segments are all contained in the lines of one ruling of
hyperbolic paraboloid.
\item
The four line segments are all contained in the lines of one ruling of
hyperboloid of one sheet.
\item
Three of the lines are concurrent.
\item
Three of the line segments lie on a plane, and the fourth either lie on the 
plane or intersect it at a point.
\item
Two of the line segments lie on a plane, the second two line segments meet the
plane at the same point.
\end{enumerate}

We present an exact, complete, and robust implementation of efficient
algorithms to compute all the lines tangent to four line segments taken from a
set of \textit{n} line segments in three dimensional Euclidean space.
The algorithm use a dual representation of a line segment, on 2D arrangement
on surface. This method was first used by M. McKenna and J. O'Rouke 
\cite{73431}. The surface is either a plane or a sphere, implemented on top of 
the \cgal{} library~\cite{cgal-home}, and are mainly based on the Arrangement
package of the library~\cite{fwh-cfpeg-04,wfzh-aptaca-05}. The results obtained
by this implementation are exact as long as the underlying number
type supports the arithmetic operations $+$, $-$, $*$, and $/$ in
unlimited precision over the algebraic and rationals. The implementation is 
complete and robust, as it handles all degenerate cases, and guarantees exact
results. We also report on the performance of our methods compared to others.

A simple method to find all the lines tangent to four lines in $\mathbb{R}^3$
using Pl\"ucker coordinates have been proposed by Hohmeyer and Teller 
\cite{TellerHohmeyer99} and also described in Redburn \cite{J-Redburn-03}. 
This method was used later by H. Everett et al. \cite{Everett08onthe} as 
predicate to line transversal.
The running time of this method is $O(n^4)$, since each quadruplets of lines
are processed.

The combinatorial complexity of all the lines tangent to four of n line
segments can be as high as $O(n^4)$. This bound is tight.
Consider the scene where the first $n / 4$ line segments are placed in the
$z = 0$ plane, and let them have equations $x = l , x=2,..., x = n/4$ from
$y = 0$ to $y = n / 4 + C$. These are vertical line segments in the $z = 0$ 
plane. Place the next $n / 4$ line segments in the $z = 0$ plane and let them
have equations $y = l, y=2,..., y = n / 4$ from $x = 0$ to $x = n / 4 + C$.
These are horizontal line segments in the $z = 0$ plane. Note that these 
$n / 2$ line segments form a grid in the $z = 0$ plane. Use the last 
$n / 2$ lines to form a grid in the $z = l$ plane. Pass a line through two 
points, one at the lower and one at the upper grid. Since each grid contains 
$n^2/16$ points the number of such lines is $n^4/256$, thus $O(n^4)$.

The implementation presented in this paper is based on a package of Cgal
called\\ \aos \cite{fwh-cfpeg-04,wfzh-aptaca-05}. It 
supports the robust construction and maintenance of arrangements of curves 
embedded on two-dimensional parametric surfaces \cite{bfhmw-scmtd-07}, and 
robust operations on them.
The implementation uses in particular arrangements of geodesic arcs embedded on
the sphere and arrangements of hyperbolas on the plane. We plan to make our new
component available as part of a future public release of Cgal as well. The
ability to robustly construct such arrangements, and carry out exact operations
on them using only (exact) rational and algebraic  arithmetic is a key property
that enables an efficient implementation.

% -----------------------------------------------------------------------------
\subsection{Related Work}
\label{ssec:related-work}
% -----------------------------------------------------------------------------
Many works have been done on finding all tangent lines to 4 geometric objects.
For a scene of $n$ triangles in $\mathbb{R}^3$ the combinatorial complexity is
$O(n^4)$ in the worst case, even when the triangles form a terrain 
\cite{Cole198911}. For $n$ disjoint convex polyhedras of constant size De Berg, 
H. Evert and J. Guibas \cite{deBerg199869} showed a lower bound of 
$\Omega(n^3)$. The upper bound is $O(n^4)$. For $K$ polytopes of total 
complexity of $n$ with $K \ll n$, when the $K$ polytopes may intersect Bronnimann
at el. \cite{Devillers06linesand} proved the tight bound of $\theta(n^{2}K^{2})$.
Marc Glisse and Sylvain Lazard \cite{1810969} showed an $\theta(n^4)$ bound for
unit balls.\newline
Hohmeyer and Teller \cite{TellerHohmeyer99} implemented using plucker coordinate
an $O(n^4)$ algorithm which finds all the tangent lines to four lines.
J. Redburn implemented an $O(n^4)$ algorithm which finds all the tangent lines 
to four triangles in  $\mathbb{R}^3$  \cite{J-Redburn-03}. Bronnimann implemented an 
$O(n^2K^2$log$n)$ time and $O(nK^2)$ space algorithm for the problem of, given 
$K$ possibly intersecting convex polyhedra, compute all tangent line segments to 
four of the polyhedras.
% -----------------------------------------------------------------------------

\label{sec:intro}
% =============================================================================

% -----------------------------------------------------------------------------
\subsection{Background and Motivation}
The interest of finding all lines tangent to four of n line segments is 
motivated by the following assembly planning problem. Given an assembly of 
rigid polyhedras, partition it into sub assemblies using two translation.
The first is a single finite translation followed by translation to infinity.
The first translation is a line segment, the critical points which represent 
the second translation are lines that tangent to the line segment and to three
polyhedras at the scene.

Other application that uses this predicate intensively is visibility problems.
In such problems, The scene is represented as a union of not necessarily 
disjoint polygonal or polyhedral objects. The  interest is in the objects that 
can be seen from a moving view point, on a line segment. Just like in the 
assembly planning problem, the line of sight is changed when it is tangent to 
3 polyhedras at the scene. I.e the critical points on the line of a moving view 
point are all the lines tangent to the line and three other line segments 
at the scene.

In both of these applications the critical events are when the lines are tangent
to the view/moving line segment and three other line segments. There are several
types of critical events, including vertex vertex, vertex edge edge and 
quadruple edge. The last event is the most significant, since it determines the
combinatorial complexity of the scene. 

\label{ssec:background}
% -----------------------------------------------------------------------------

\subsection{Outline}
\label{ssec:outline}

The rest of this paper is organized as follows. Preliminaries mathematical 
properties required to the implementation is described in Section 2 along with
the necessary terms and definitions. In
Section 3 we provide implementation details of the general case. Section 4 
presents all the degenerate cases that are not discussed in the preceding 
sections. We report on experimental results in Section 5.

% -----------------------------------------------------------------------------

% =============================================================================
\section{Terms and Definitions or Preliminaries}
\label{sec:terms}
In this section we review some basic mathematical properties of lines in space.
\\

It is well known that all the lines tangent to three disjoint skew lines in 
$\mathbb{R}^3$ are on one ruling of a ruled surface. This surface is either 
hyperbolic paraboloid or hyperboloid of one sheet. See figure 
\ref{fig:paraboloid}. All the lines on the other ruling of one of these ruled 
surfaces are tangent to the three lines.
A fourth line can be on the same ruling, in that case there are infinite
lines tangent to the four lines. Or stab the surface in 0, 1 or 2 
points, in such case there will be respectively 0, 1 or 2 lines  
tangent to the four lines.

\begin{figure}[t]
\begin{center}
    \begin{tabular}{c c}
        \psfig{figure=Fig/Hyperbolic-paraboloid.eps,width=2.5in,silent=} ~~~&~~~
        \psfig{figure=Fig/Hyp_of_one_sheet.eps,width=2.5in,silent=}
    \end{tabular}
\end{center}
\vspace{-2ex}
\caption{{\sf The left figure is hyperbolic paraboloid and the right is 
hyperboloid of one sheet. All the lines on the other ruling of one of these 
ruled surfaces is tangent to three lines.
}} \label{fig:paraboloid}
\end{figure}


\begin{lemma}
\label{OneDegreeOfFreedom}
A line tangent to 3 skew line segments $L^1$, $L^2$ and $L^3$, has one degree of 
freedom, and it is represented as hyperbola at the parameter space defined by 
$L^1$ and $L^2$.
\end{lemma}

\begin{proof}
Given three lines $L_1$, $L_2$ and $L_3$ on $\mathbb{R}^3$:
\[
{(L^1_{x1} + L^1_{t}L^1_{x2}, L^1_{y1} + L^1_{t}L^1_{y2}, L^1_{z1} + 
L^1_{t}L^1_{z2}), 0 \leq L^1_t \leq 1}
\]
\[
{(L^2_{x1} + L^2_{t}L^2_{x2}, L^2_{y1} + L^2_{t}L^2_{y2}, L^2_{z1} + 
L^2_{t}L^2_{z2}), 0 \leq L^2_t \leq 1}
\]
\[
{(L^3_{x1} + L^3_{t}L^3_{x2}, L^3_{y1} + L^3_{t}L^3_{y2}, L^3_{z1} + 
L^3_{t}L^3_{z2}), 0 \leq L^3_t \leq 1}
\]
\newline
A line $L$ tangent to $L^1$, $L^2$ and $L^3$ must fulfill the following equations:
\[
(1-t)(L^1_{x1} + L^1_{t}  L^1_{x2}) + t(L^2_{x1} + L^2_{t}  L^2_{x2}) = 
(L^3_{x1} + L^3_{t}  L^3_{x2})
\]
\[
(1-t)(L^1_{y1} + L^1_{t}  L^1_{y2}) + t(L^2_{y1} + L^2_{t}  L^2_{y2}) = 
(L^3_{y1} + L^3_{t}  L^3_{y2})
\]
\[
(1-t)(L^1_{z1} + L^1_{t}  L^1_{z2}) + t(L^2_{z1} + L^2_{t}  L^2_{z2}) = 
(L^3_{z1} + L^3_{t}  L^3_{z2})
\]
\newline
From the development of the equations we get the following hyperbola:
\[
H:\>\>\>\>\>\>\>\>\>\> (L^2_{t} - k)(L^1_t - h) = m
\]
$L^2_t$ represents the y-axis and $L^1_t$ represents the x-axis at the parameter 
space. Each point on the parameter space represent a line that is tangent to
$L^1$ and $L^2$, the points in this hyperbola represents the lines in 3-space 
that intersect $L_1$, $L_2$ and $L_3$.
\end{proof}

Two such hyperbolas may intersect in either 0, 1 ,2 points or infinite,
when the hyperbolas overlap. Respectively we will get 0, 1, 2 or infinite 
tangent lines to the four lines.
% =============================================================================

% =============================================================================
\section{The General Case}
\label{sec:general}
The input to the program is a set $S=\{s_1,\ldots,s_n\}$ of $n$ line segments.
\newline
Given 3 disjoint skew line segments in $\mathbb{R}^3$ euclidean space $S_1$, $S_2$, and 
$S_i$, a line L tangent to these 3 line segments has 1 degree of freedom.\newline
We cast the problem into two dimensional parameter plane, in which the supporting
line of $S_1$ is parametrized to the x-axis, the supporting line of $S_2$ is
parametrized to the y-axis, and $S_i$ is represented by a bounded hyperbola on 
the parameter plane.
Since $S_1$ and $S_2$ are line segments, only the square ([0,0] [1,1]) is 
relevant. Each point on the square represents a tangent line to $S_1$ and $S_2$.
Each point on the hyperbola inside the square ([0,0], [1,1]) represents a
tangent line to $S_1$, $S_2$ and $S_i$.

Each two line segments $S_i$, $S_j$, $j>i$ are parametrized to an arrangement on
plane. For each line segment $S_k$ $k>j$  a bounded hyperbola is added to the
arrangement. The intersection points between these hyperbolas inside the bounded
square represent tangent lines to 4 segments.

\begin{figure}[t]
\begin{center}
    \begin{tabular}{c c}
        \psfig{figure=Fig/arr_on_plane.eps,width=2.5in,silent=} ~~~&~~~
        \psfig{figure=Fig/two_triangles.eps,width=2.5in,silent=}
    \end{tabular}
\end{center}
\vspace{-2ex}
\caption{{\sf The first arrangement was created by five segments, four co-planer
and the last, crosses the plane. 
The second arrangement was created by eight segments, where the first two are skew
and the other six form 2 triangles. Each 3D triangle mapped to general triangle at
the arrangement.
}} \label{fig:arr_on_plane}
\end{figure}

% =============================================================================

% =============================================================================
\section{Degenerate Cases}
\label{sec:degenerate}
When $S_i$ and $S_j$ intersect 2 parameter spaces are used.
\begin{enumerate}
\item A sphere - The sphere origin is the intersection point of $S_i$ and $S_j$.
Each point on the sphere represents a tangent line to the intersection point of
$S_i$ and $S_j$. A segment $S_k$, is parametrized as 2 bounded arcs on the 
sphere. Each point on the arc represents a tangent line to $S_i$, $S_j$, and
$S_k$.  Since the arcs are symmetric, only the upper half of the sphere is used
by the algorithm.\newline
For each line segment $S_k$ one or two arcs are added to the upper half of the
sphere arrangement. The intersection points between these arcs represent a
tangent line to 2 segments and the intersection point of $S_i$ and $S_j$.\\
When the supporting line of segment $S_i$ is crossing the sphere origin or
when $S_i$ is a point, a point is added to the sphere. If the point ``falls''
on geodesic arc or another point on the sphere, it represents a line that passes
through four line segments.\newline
Additional degeneracy occurs when a line segment contains the origin of the sphere.
In this case all the point on the different geometric objects on the sphere
(geodesic arcs and points) represent lines through four line segments. When two
line segments $S_i$ and $S_j$ contain the sphere origin, I.e $S_1$, $S_2$, $S_i$
and $S_j$ are concurrent, all of the points on the sphere represent line segments
that passes through four line segments (the origin) and the origin is the output.

%%TODO picture of a sphere (can take the example of the rectangle)
\item
A plane - Since $S_i$ and $S_j$ intersect, they are lying on the same plane P. 
A tangent line to $S_i$, $S_j$ must lie on P. Hence, for each line segment $S_k$
only the intersection objects with P is relevant.
If $S_k$ intersect with P at a point, a bounded hyperbola or a line segment is 
added to the plane arrangement. Otherwise $S_k$ is contained at P, and up to 
four faces are added to the arrangement, each point on the face represents a tangent 
line to $S_i$, $S_j$ and $S_k$.
When 2 faces created by two different line segments $S_k$, $S_l$ overlap, the 
overlapping face is returned to the output container as bounded polygon. 
At this case infinite lines passes through to $S_i$, $S_j$, $S_k$ and $S_l$. 
See the pink faces at figure \ref{fig:arr_on_plane}.
At each face a counter of the number of overlap faces is maintained. The underlie 
curve of each edge between two faces with counters $\geq 1$, also represents 
infinite lines that passes through 4 or more segments. 
See blue curves at figure \ref{fig:arr_on_plane}.

\end{enumerate}

When $S_i$ and $S_j$ are co-planer, I.e $S_i$, and $S_j$ are parallel, or the 
supporting lines of $S_i$, and $S_j$ intersect. An arrangement on plane is 
created, just as the second arrangement when $S_i$ and $S_j$ are intersecting. 
This is also true when either $S_1$ or $S_2$ are points.\\
When both $S_1$ and $S_2$ are points the line that connects $S_1$ and $S_2$ is
the only relevant line. In this case we only compute the intersection of this 
line with the other line segments and verify if it tangent to additional 2 line
segments.

When four line segments overlap, the common line segment to the fourth is the
output.\\

Theoretical bounds for the algorithm are $O((n^3 + P)log n)$ running time, and 
$O(n^2 + P)$ working storage. Where n is the input size and $P$ is the output 
size($P$ is bounded by $n^4$).
% =============================================================================

% =============================================================================
\section{Experimental Results}
Our program can handle all inputs. However, other implementations can not.
Hence  we  limit ourselves to a small set of test cases of input in general 
position, where we compare the impact of the output size on the process time
consumption. The results listed in \ref{table:results} were produced by 
experiments conducted on a Pentium PC clocked at 1.7 GHz (TBD). 
The first set of input is random sets line segments. The second is short line
segments and the third is long line segments.
The last input is perturbated 2 grids of line segments as described at the
introduction.

\begin{table}[ht] 
\caption{} % title of Table 
\begin{tabular}{p{2cm} | p{3cm} | p{2cm} | p{3cm} | p{2cm}} % centered columns (4 columns) 
\hline\hline %inserts double horizontal lines
Num Of Segments & Conic Traits & Rational Arc Traits & Quad Tangent (Work of J. Redburn) & Num Of Tangent Lines\\
%heading
\hline % inserts single horizontal line 
10-rand & 0.04800 & 0.03200 & 0.34802 & 15\\
\hline % inserts single horizontal line 
30-rand & 4.50000 & 3.10000 & 49.04710 & 2620\\ 
\hline % inserts single horizontal line 
35-rand & 4.50000 & 3.10000 & 49.04710 & 2620\\
\hline % inserts single horizontal line 
40-rand & 7.04000 & 8.90000 & 165.59800 & 8746\\
\hline % inserts single horizontal line 
50-rand & 38.43000 & 20.52000 & 373.203 & 20742\\
\hline % inserts single horizontal line 
10-short & 0.02800 & 0.02400 & 0.33602 & 0\\
\hline % inserts single horizontal line 
30-short & 0.76805 & 0.70004 & 46.45890 & 0\\
\hline % inserts single horizontal line 
35-short & 1.34000 & 1.19207 & 85.79340 & 0\\
\hline % inserts single horizontal line 
40-short & 1.77211 & 1.78411 & 145.97300 & 0\\
\hline % inserts single horizontal line 
50-short & 3.32021 & 3.47222 & 353.47 & 0\\
\hline % inserts single horizontal line 
10-long & 0.11200 & 0.05200 & 0.38002 & 67\\
\hline % inserts single horizontal line 
30-long & 9.31800 & 3.58422 & 47.61900 & 5862\\
\hline % inserts single horizontal line 
35-long & 15.82500 & 8.94456 & 90.60570 & 9133\\
\hline % inserts single horizontal line 
40-long & 129.856 & 60.69580 & 389.91200 & 80016\\
\hline % inserts single horizontal line 
40-grid & 92.6938 & 19.50120 & 181.66300 & 12264\\
% [1ex] adds vertical space
\hline %inserts single line 
\end{tabular} 
\label{table:results} % is used to refer this table in the text 
\end{table}
\label{sec:experimental-results}
% =============================================================================

% =============================================================================
\bibliography{abrev-short,lines_thru_segs}
\bibliographystyle{abbrv}
% =============================================================================
\appendix
% =============================================================================
% =============================================================================
\end{document}
